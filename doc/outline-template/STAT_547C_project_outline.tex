%%%%%%%%%%%%%%%%%%%%%%%%%%%%%%%%%%%%%%%%%%%%%%%%%%%%%%%%%%%%%%%%%%%%%%%%%%%%%%%%%%%%
% Template for STAT 547C Final Project Outline
% Author: Ben Bloem-Reddy <benbr@stat.ubc.ca>
% Date: Oct. 17, 2019
% Acknowledgments: ETH, Peter Orbanz, John Cunningham
%%%%%%%%%%%%%%%%%%%%%%%%%%%%%%%%%%%%%%%%%%%%%%%%%%%%%%%%%%%%%%%%%%%%%%%%%%%%%%%%%%%%

\documentclass[]{STAT_547C}
\usepackage{STAT_547C}
% NOTE: change the name and email address to your name in STAT_547C.sty

\usepackage{booktabs}
\usepackage{amsmath,amsthm,amssymb,amsfonts}

\usepackage[sorting=none,backend=biber,bibstyle=alphabetic,citestyle=alphabetic,giveninits=true,natbib=true]{biblatex}
\bibliography{../../ref/STAT_547C.bib} % add the title and location of your bibliography file

\begin{document}

% NOTE: You will replace the title below with your actual Title.
\makeGenericHeader{Exchangeability in Gaussian Process Regression and\\ Bayesian Optimization}{Project Outline}
\vspace{-2cm}


%%%%%%%%%%%%%%%%%%%
\section{Title}

The working title of my project is \emph{Exchangeability in Gaussian Process Regression and Bayesian Optimization}.  

%%%%%%%%%%%%%%%%%%%
\section{Background}


Exchangeability is a fundamental---although oftentimes overlooked---concept in probability which has elegant theoretical underlying results. Furthermore, the number of potential uses of exchangeability in contemporary statistics seems to be growing. For instance, \cite{Kingman:1978, Aldous:2010, Orbanz:Roy:2015} discuss applications of exchangeability on a plethora of subjects, from population genetics to general random structures. \\

Another area (?) in Bayesian computational statistics that has enjoyed a surge in popularity is the use of Gaussian Process regression for modelling complex phenomena. Bayesian Optimization is one application of this technique which has grown substantially in the recent years. \\

In \cite{McCullagh:2005:ExchAndReg}, McCullah discusses exchangeability in the context of regression models. I am interested in giving a more detailed treatment of these ideas in the case of Gaussian Process regression, specifically studying it in a Bayesian framework.



%%%%%%%%%%%%%%%%%%%
\section{Technical aspects}

The project will draw on technical aspects of the following areas: independence, conditioning and disintegration, stochastic processes, optimization theory.


%%%%%%%%%%%%%%%%%%%
\section{Literature}

The key references for this project are:

\begin{itemize}
  \item \citet{McCullagh:2005:ExchAndReg}, as mentioned above.
  \item \cite{Kingman:1978, Bernardo:1996:Exch} give a probabilistic treatment of exchangeability; \cite{deFinetti:1930:RepTheorem} is the original paper in which de Finetti proves his integral representation theorem.
  \item 
\end{itemize}


%%%%%%%%%%%%%%%%%%%
\section{Plan}

I will carry out this project with the following sequence of steps: 
\begin{enumerate}
  \item I will map ideas presented in a statistical theoretical framework in the papers of \citet{Dawid:1979:CondIndStatTheory,Dawid:1980:CondIndStatOp} to one that takes into account computational requirements of  inference algorithms, with a focus on sufficiency and adequacy.
  \item I will review methods from computational statistics and machine learning that rely on conditional independence. 
  \item I will focus on one of these methods for an in-depth study of the effects of conditional independence on computational complexity.
\end{enumerate}


%%%%%%%%%%%%%%%%%%%
\section{Why I'm interested in this topic}

I am interested in both Gaussian Process regression and its applications and the concept of exchangeability, and this project seems like a nice way of joining them.


%%%%%%%%%%%%%%%%%%%
\printbibliography


\end{document}

