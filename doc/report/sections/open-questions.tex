% !TEX root = ../main.tex

% open questions section

\section{Discussion} \label{section:discussion}


The main question posed by this work was whether exchangeability could have any role in dependence modelling, and especially in Gaussian process regression. Proposition \ref{prop:exchangeability} implies that de Finetti's classical representation theorem cannot be used in such a setting. It is easy to conclude the same for other dependence modelling settings, such as regression analysis. 
\\


However, GP regression does satisfy more general notions of exchangeability: local and regression exchangeability for all GPs, and partial exchangeability for noisy GPs. Furthermore, by Propositions \ref{prop:GP_partial_exch} and \ref{prop:GP_local_exchangeability} we know that any noisy GP can be represented as a conditional independence sequence as in Theorems \ref{thm:definetti_partial} and \ref{thm:de_finetti_local}, respectively. Proposition \ref{prop:GP_NF_local_exchangeable} guarantees such a representation for noise-free processes too. This is summarised in Table \ref{table:summary}. It is also clear that other dependence models also satisfy some of these notions of exchangeability. For example, the proof of Proposition \ref{prop:GP_reg_exch} can be easily adapted for linear regression analysis and ANOVA models.
\\






\begin{table}[h]
\centering
\begin{tabular}{l|cc|cc}
                & \multicolumn{2}{c|}{Noise-free}                                              & \multicolumn{2}{c}{Noisy}                                                     \\  
Exchangeability & \begin{tabular}[c]{@{}c@{}}Satisfies\\ definition\end{tabular} & \begin{tabular}[c]{@{}c@{}}Representation\\ theorem\end{tabular} & \begin{tabular}[c]{@{}c@{}}Satisfies\\ definition\end{tabular} & \begin{tabular}[c]{@{}c@{}}Representation \\ theorem\end{tabular} \\ \hline
Classical       & No        & No                                                               & No        & No                                                                \\
Partial         & No        & No                                                               & Yes       & Yes                                                               \\
Local           & Yes       & Yes                                                              & Yes       & Yes                                                               \\
Regression      & Yes       & No                                                               & Yes       & No                                                               
\end{tabular}
\caption{Summary of findings for GP regression and exchangeability. For each notion of exchangeability, does GP regression (either noise-free or noisy) satisfy the definition and, if so, are there any representation theorems available?}
\label{table:summary}
\end{table}




There seems to be a trade-off between adequacy and theoretical power in these notions of exchangeability: partial exchangeability is generally easy to verify and has a de Finetti-like representation theorem, but settings in which no replicates---and a countable number of them at that---are available are never going to exhibit partial exchangeability. Regression exchangeability is also easy to verify and, furthermore, satisfied by more general processes, but it lacks a representation theorem. Lastly, local exchangeability is a more powerful theoretical tool than both, but it is still unclear what processes exhibit this characteristic. It would appear that many do, but this remains to be proved. We expand these points in the research directions section below.




\subsection{Open questions and research directions}


From those exchangeability notions studied here, local exchangeability appears to be the most flexible: many real life processes naturally exhibit \textit{near} exchangeability. The fact that de Finetti's result proves robust to perturbations of exact exchangeability is remarkable. 
\\


However, as we hope to have made evident in this work, proving that a certain process is locally exchangeable is a complicated task at best---even with the sufficiency conditions of Proposition \ref{prop:sufficient_local_exchangeability}. \cite{CampbellEtAl:2019:LocalExch} prove that some popular Bayesian nonparametric models are locally exchangeable, but much work remains to be done in this respect. A first approach would be to simply prove, model by model, whether a certain process is locally exchangeable or not, and under what conditions, until a sufficiently large repertoire of locally exchangeable processes is available. 
\\

Nonetheless, a more general result would be ideal, both from a theoretical and a practical point of view. It seems that the task of finding a suitable function $f$ is what tends to complicate matters, and so it would be desirable to obtain an existence-type theorem. That is, perhaps the question to be asked is not \textit{what} $f$ works, but under what conditions such an $f$ exists, which would then exhibit a process as locally exchangeable without having to manually compute $f$. We pursue this point a little bit further. \cite{CampbellEtAl:2019:LocalExch} show that $f$---specifically its rate of decay---plays a role in controlling the smoothness of local exchangeability. Perhaps an ideal result would allow us to determine not only if an $f$ that renders a process $f$-locally exchangeable exists, but also its rate of decay.
\\


Lastly, regression exchangeability is also easy to verify for any given sequence and would seem to be an ideal setting for most of dependence modelling---at the very least for regression analysis. It is, however, without a representation theorem, and thus determining if one exists would make it a much more valuable notion. We conjecture how such a result would look.
\\


Theorem \ref{thm:definetti_partial}---partial exchangeability representation theorem---essentially says that the classes of a partially exchangeable process exhibit conditional independence (although not i.i.d. behaviour, as in classical exchangeability). A result for regression exchangeability can be expected to capture the same idea, although instead of classes we would have samples with same covariate values. In other words, if a partially exchangeable sequence can be arranged by its exchangeable classes, a regression exchangeable one can be arranged by its samples. Hence, for such a process $(X_t)_{t \in \calT}$ we define, for every finite $T \subset \calT$, the set of samples equivalent to $T$:
\begin{equation*}
	\calE (T) = \{ T' \subset \calT \, : \, T' = T \},
\end{equation*}
Condition \ref{cond:reg_exch_2} in Definition \ref{def:regression_exchangeability} can be rewritten as $T' \in \calE(T) \implies X_T \equdist X_{T'}$. Observe that this specifies equivalence classes in the covariates: $T' \in \calE (T)$ if and only if $T \in \calE (T')$. Let $d$ be the number of such classes ($d$ may very well be $\infty$). We would expect these classes to be conditionally independent given some measure $\mu$ on $\calP^d$, where again $\calP$ is the set of probability measures in $(\calX, \calB)$. This leads us to the following conjecture.

\begin{conjecture} \label{conjecture:definetti_regression}
	The sequence $(X_t)_{t \in \calT}$ is regression exchangeable if and only if there exists a unique probability measure $\mu$ on $\calP^d$ such that, for all finite subsets $T_1, ..., T_d$ of each equivalence class, and for all $A_1, ..., A_d$ with $A_i \in \calB^{|T_i|}$,
	\begin{equation} \label{eq:definetti_regression_thm}
		\bbP \left\{ (X_{T_i}) \in A_i \, : \, i=1, ..., d  \right\} = \int_{\calP^d}  \prod_{i=1}^d F_i(A_i) \, \mu(dF_1, ..., dF_d).
	\end{equation}
\end{conjecture}

The right-hand side of Equation (\ref{eq:definetti_regression_thm}) may seem exactly the same as that in Theorem \ref{thm:definetti_partial}, but their meanings are inherently different: the former expresses exchangeability between samples, and the latter between classes. Proving Conjecture \ref{conjecture:definetti_regression}, or a suitable modification of it, may lead to a better understanding of the role of exchangeability in dependence modelling.

























