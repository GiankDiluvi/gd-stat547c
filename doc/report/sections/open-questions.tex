% !TEX root = ../main.tex

% open questions section

\section{Discussion} \label{section:discussion}


The main question posed by this work was whether exchangeability could have any role in dependence modelling, and especially in Gaussian process regression. Proposition \ref{prop:exchangeability} implies that de Finetti's classical representation theorem cannot be used in such a setting. It is easy to conclude the same for other dependence modelling settings, such as regression analysis. 
\\


However, GP regression does satisfy more general notions of exchangeability: local and regression exchangeability for all GPs, and partial exchangeability for noisy GPs. It is clear that other dependence models also satisfy some of them. For example, the proof of Proposition \ref{prop:GP_reg_exch} can be easily adapted for linear regression analysis and ANOVA models. Furthermore, by Propositions \ref{prop:GP_partial_exch} and \ref{prop:GP_local_exchangeability} we know that any noisy GP can be represented as a conditional independence sequence as in Theorems \ref{thm:definetti_partial} and \ref{thm:definetti_partial}, respectively. Proposition \ref{prop:GP_NF_local_exchangeable} guarantees such a representation for noise-free processes. This is summarised in Table \ref{table:summary}.



\begin{table}[h]
\centering
\begin{tabular}{l|cc|cc}
                & \multicolumn{2}{c|}{Noise-free}                                              & \multicolumn{2}{c}{Noisy}                                                     \\  
Exchangeability & \begin{tabular}[c]{@{}c@{}}Satisfies\\ definition\end{tabular} & \begin{tabular}[c]{@{}c@{}}Representation\\ theorem\end{tabular} & \begin{tabular}[c]{@{}c@{}}Satisfies\\ definition\end{tabular} & \begin{tabular}[c]{@{}c@{}}Representation \\ theorem\end{tabular} \\ \hline
Classical       & No        & No                                                               & No        & No                                                                \\
Partial         & No        & No                                                               & Yes       & Yes                                                               \\
Local           & Yes       & Yes                                                              & Yes       & Yes                                                               \\
Regression      & Yes       & No                                                               & Yes       & No                                                               
\end{tabular}
\caption{Summary of findings for GP regression and exchangeability. For each notion of exchangeability, does GP regression (either noise-free or noisy) satisfy the definition and, if so, are there any representation theorems available?}
\label{table:summary}
\end{table}







\subsection{Open questions and research directions}


Observe that partial exchangeability is generally easy to verify and does have a representation theorem. However, it is clear that settings in which no replicates---and a countable number of them at that---are available are never going to exhibit partial exchangeability. In this sense, local exchangeability appears to be the most flexible: many real life situations naturally exhibit \textit{near} exchangeability. The fact that de Finetti's result proves robust to perturbations of exact exchangeability is remarkable. 
\\


However, as we hope to have made evident in this work, proving that a certain process is locally exchangeable is a complicated task at best---even with the sufficiency conditions of Proposition \ref{prop:sufficient_local_exchangeability}. \cite{CampbellEtAl:2019:LocalExch} prove that some popular Bayesian nonparametric models are locally exchangeable, but much work remains to be done in this respect. 
\\

A first approach would be to simply prove, model by model, whether a certain process is locally echangeable or not, and under what conditions, until a sufficiently large repertoire of locally exchangeable processes is available. However, a more general result would be ideal, both from a theoretical and a practical point of view. 
\\

It seems that the task of finding a suitable function $f$ is what tends to complicate matters, and so it would be desirable to obtain an existence-type theorem. That is, perhaps the question to be asked is not \textit{what} $f$ works, but under what conditions such an $f$ exists, which would then exhibit a process as locally exchangeable without having to manually compute $f$. 
\\

We pursue this point a little bit further. \cite{CampbellEtAl:2019:LocalExch} show that $f$---specifically its rate of decay---plays a role in controling the smoothness of local exchangeability. Perhaps an ideal result would allow us to know not only if an $f$ that renders a process $f$-locally exchangeable exists, but also its rate of decay.
\\


Lastly, regression exchangeability is also easy to verify for any given sequence and would seem to be an ideal setting for most of dependence modelling---at the very least for any regression model. It is however without a representation theorem, and thus determining if one exists would make it a much more vluable notion. In this spirit, we conjecture how such a result would look.